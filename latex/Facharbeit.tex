\documentclass[12pt, a4paper]{report}
\usepackage[utf8]{inputenc}
\usepackage{amsmath, amsfonts, amsthm}
\usepackage[ngerman]{babel}
\usepackage{xcolor}
\usepackage{geometry}
\geometry{
 a4paper,
 total={170mm,257mm},
 left=30mm,
 right=30mm,
 top=20mm,
 bottom=40mm
 }

\usepackage[framemethod=tikz]{mdframed}

% Proposition zur korrekten Nummerierung
\newmdtheoremenv[
]{prop}{Proposition}[section]

% Spezieller Style für die Definitionen
\theoremstyle{definition}
\newmdtheoremenv[
  hidealllines=true,
  leftline=true,
  innerleftmargin=10pt,
  innerrightmargin=10pt,
  innertopmargin=0pt,
]{defi}[prop]{Definition}

% Bemerkungen Theorem
\newmdtheoremenv[
  hidealllines=true,
  innertopmargin=0pt,
]{rem}[prop]{Bemerkung}

% Korollar Theorem
\newmdtheoremenv[
  hidealllines=true,
  innertopmargin=0pt,
]{kor}[prop]{Korollar}

% Lemma Theorem
\newmdtheoremenv[
  hidealllines=true,
  innertopmargin=0pt,
]{lem}[prop]{Lemma}

\makeatletter

\newcommand*{\rom}[1]{\expandafter\@slowromancap\romannumeral #1@}
\newcommand\omicron{o}


\makeatother
\title{\LARGE   Facharbeit im Leistungskurs Mathematik  \vspace*{0.5cm} \\
  \Huge Renditenverteilung am Kapitalmarkt}
\LARGE \author{Jesse Georgias}
\LARGE \date{März 2021}

\begin{document}

\maketitle
\newpage
\vspace*{\fill}
\begin{normalsize}
\noindent
  \textbf{Satz:}
  \LaTeX
\end{normalsize}
\vspace*{5pt}
\hrule
\vspace{5pt}
\noindent
\textbf{Quellen:}
\\
Christian Limbacher, \textit{Schiefe und Kurtosis von Aktienrenditen}, Masterarbeit, Johannes Kepler Universität Linz, 2018 \\
Benoit Mandelbrot, \textit{The Variation of certain speculative Prices}, The University of Chigago Press, 1963 \\
Hanspeter Schmidli, \textit{Einführung in die Stochastik}, Vorlesungsnotizen, Universität zu Köln, 2019
\tableofcontents
\newpage
\chapter{Einleitung}
Von dem Wetter über Fußballergebnisse bis zur Verspätung der Bahn, unser Alltag wird maßgeblich von Ereignissen geprägt auf die wir so gut wie keinen Einfluss nehmen können und die ebenso schwer vorherzusagen sind. In dieser Arbeit werde ich mich mit einer der Mysteriösesten dieser Zufallsgrößen beschäftigen: den Finanzmärkten.
\section{Zielsetzung}
Auf den ersten Blick sind die Vorgänge in den Finanzmärkten das reinste Chaos. Niemand kann zuverlässig vorhersagen wie sich die Märkte morgen verhalten werden. Mein Ziel ist es, in dieses Chaos Struktur zu bringen und somit dem Leser zu einem besserem Verständniss, besonders von den assoziierten Risiken, zu verhelfen. Weiterhin soll diese Arbeit aufzeigen, wie mathematische Theorien in der realen Welt Anwendung finden und somit den essenziellen Realitätsbezug eines als theoretisch wahrgenommenen Faches etablieren.


\section{Wahrscheinlichkeitsräume}

\chapter{Die Gaussche Normalverteilung}









\end{document}
