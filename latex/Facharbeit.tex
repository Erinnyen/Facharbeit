\documentclass[12pt, a4paper]{report}
\usepackage[utf8]{inputenc}
\usepackage{amsmath, amsfonts, amsthm}
\usepackage[ngerman]{babel}
\usepackage{xcolor}
\usepackage{geometry}
\geometry{
 a4paper,
 total={170mm,257mm},
 left=30mm,
 right=30mm,
 top=20mm,
 bottom=40mm
 }

\usepackage[framemethod=tikz]{mdframed}

% Proposition zur korrekten Nummerierung
\newmdtheoremenv[
]{prop}{Proposition}[section]

% Spezieller Style für die Definitionen
\theoremstyle{definition}
\newmdtheoremenv[
  hidealllines=true,
  leftline=true,
  innerleftmargin=10pt,
  innerrightmargin=10pt,
  innertopmargin=0pt,
]{defi}[prop]{Definition}

% Bemerkungen Theorem
\newmdtheoremenv[
  hidealllines=true,
  innertopmargin=0pt,
]{rem}[prop]{Bemerkung}

% Korollar Theorem
\newmdtheoremenv[
  hidealllines=true,
  innertopmargin=0pt,
]{kor}[prop]{Korollar}

% Lemma Theorem
\newmdtheoremenv[
  hidealllines=true,
  innertopmargin=0pt,
]{lem}[prop]{Lemma}

\makeatletter

\newcommand*{\rom}[1]{\expandafter\@slowromancap\romannumeral #1@}
\newcommand\omicron{o}


\makeatother
\title{\LARGE   Facharbeit im Leistungskurs Mathematik  \vspace*{0.5cm} \\
  \Huge Renditenverteilung am Kapitalmarkt}
\LARGE \author{Jesse Georgias}
\LARGE \date{März 2021}

\begin{document}

\maketitle
\newpage
\vspace*{\fill}
\begin{normalsize}
\noindent
  \textbf{Satz:}
  \LaTeX
\end{normalsize}
\vspace*{5pt}
\hrule
\vspace{5pt}
\noindent
\textbf{Quellen:}
\\
Christian Limbacher, \textit{Schiefe und Kurtosis von Aktienrenditen}, Masterarbeit, Johannes Kepler Universität Linz, 2018 \\
Benoit Mandelbrot, \textit{The Variation of certain speculative Prices}, The University of Chigago Press, 1963 \\
Hanspeter Schmidli, \textit{Einführung in die Stochastik}, Vorlesungsnotizen, Universität zu Köln, 2019
\tableofcontents
\newpage
\chapter{Einleitung}
\section{Einleitung}








\end{document}
