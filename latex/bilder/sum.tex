\documentclass[12pt, a4paper]{report}
\usepackage[utf8]{inputenc}
\usepackage{amsmath, amsfonts, amsthm}
\usepackage[ngerman]{babel}
\usepackage{xcolor}
\usepackage{geometry}
\geometry{
 a4paper,
 total={170mm,257mm},
 left=30mm,
 right=30mm,
 top=20mm,
 bottom=40mm
 }
 
\usepackage[framemethod=tikz]{mdframed}
 
% Proposition zur korrekten Nummerierung
\newmdtheoremenv[
]{prop}{Proposition}[section]

% Spezieller Style für die Definitionen
\theoremstyle{definition}
\newmdtheoremenv[
  hidealllines=true,
  leftline=true,
  innerleftmargin=10pt,
  innerrightmargin=10pt,
  innertopmargin=0pt,
]{defi}[prop]{Definition}

% Bemerkungen Theorem
\newmdtheoremenv[
  hidealllines=true,
  innertopmargin=0pt,
]{rem}[prop]{Bemerkung}

% Korollar Theorem
\newmdtheoremenv[
  hidealllines=true,
  innertopmargin=0pt,
]{kor}[prop]{Korollar}

% Lemma Theorem
\newmdtheoremenv[
  hidealllines=true,
  innertopmargin=0pt,
]{lem}[prop]{Lemma}

\makeatletter

\newcommand*{\rom}[1]{\expandafter\@slowromancap\romannumeral #1@}
\newcommand\omicron{o}


\makeatother
\title{\LARGE   Zusammefassung der Vorlesung  \vspace*{0.5cm} \\
  \Huge Lineare Algebra I}
\LARGE \author{Jesse Georgias}
\LARGE \date{Dezember 2021}

\begin{document}

\maketitle
\newpage
\vspace*{\fill}
\begin{normalsize}
\noindent
  \textbf{Satz:}
  \LaTeX
\end{normalsize}
\vspace*{5pt}
\hrule
\vspace{5pt}
\noindent
\textbf{Quellen:}
\\
Gerd Fischer, \textit{Lineare Algebra I}, Grundkurs Mathematik, Springer Spektrum, 2020
Sander Zwegers, \textit{Lineare Algebra I}, Vorlesungsnotizen, Universität zu Köln, 2021
\tableofcontents
\newpage
\chapter{Lineare Gleichungssysteme}
\section{Lineare Gleichungssysteme}


\chapter{Matrizenoperationen}

\chapter{Determinante}

\chapter{Vektorräume}

\chapter{Lineare Abbildungen}
\section{Lineare Abbildungen}

\begin{defi}
  $\varphi:V\longrightarrow W$ ist eine Lineare Abbildung (\textit{Homomorphismus}) wenn:\vspace*{0.3cm} \\
    \indent (a) $ \varphi(v + w) = \varphi(v) + \varphi(w) $ \indent $\forall v, w \in V$ \vspace*{0.2cm} \\
    \indent (b) $ \varphi(\lambda v) = \lambda \varphi(v) $ \indent $ \forall \lambda \in \mathbb{K}, \ 
    \forall v \in V$
\end{defi} 

\begin{kor}
  $\lambda = 0 \Rightarrow \varphi(0v) = 0\varphi(v) \Rightarrow \varphi(0) = 0$
\end{kor}

\begin{defi}
  $Hom_{\mathbb{K}}(V, W)$ ist die Menge aller \textit{Homomorphismen} von $V$ nach $W$. Somit
  $\varphi \in Hom_{\mathbb{K}}(V, W)$. Weiterhin heißt $\varphi$: \vspace*{0.2cm} \\
    \indent (a) \textit{Monomorphismus}, wenn $\varphi$ injektiv ist; \vspace*{0.1cm} \\
    \indent (a) \textit{Endomorphismus}, wenn $\varphi$ surjektiv ist; \vspace*{0.1cm} \\
    \indent (a) \textit{Isomorphismus}, wenn $\varphi$ bijektiv ist.
\end{defi}

\begin{rem}
  Eigenschaften von Abbildungen: \vspace*{0.3cm}\\
    \indent (a) \textit{Injektivität}: Jedes Urbild hat höchstens ein Abbild \vspace*{0.2cm} \\
    \indent (b) \textit{Surjektivität}: Jedes Abbild hat ein nicht-leeres Urbild \vspace*{0.2cm} \\
    \indent (c) \textit{Bijektivität}: Injektivität und Surjektivität sind gegeben.
\end{rem}

\begin{defi}
  \textit{Bild} und \textit{Kern}: \vspace*{0.2cm} \\
    \indent $Im(\varphi) := \varphi(V) = \{ \varphi(v) \ | \ v \in V\} \subset W$ \vspace*{0.2cm}\\
    \indent$Ker(\varphi) := \{v \in V \ | \ \varphi(v) = 0\} \subset V$ 
\end{defi}

\begin{lem}
  Wenn $\varphi$ \textit{injektiv} ist, so gilt $Ker(\varphi)=\{0 \}$ (Da andernfalls mehrere Urbilder    das    selbe Abbild hätten). \\
Angenommen $Ker(\varphi)=\{0 \} \text{ und } \varphi(v) = \varphi(\tilde{v})$, so folgt: \\
  \indent $\Rightarrow \varphi(v - \tilde{v}) = \varphi(v) - \varphi(\tilde{v}) = 0 \text{ und } v - \tilde{v} \in Ker(\varphi)$ \\
  \indent $\Rightarrow v - \tilde{v} = 0 \Rightarrow v = \tilde{v} \Rightarrow \varphi \text{ ist \textit{injektiv}.}$ \\
  \indent $\Rightarrow \text{ Wenn } Ker(\varphi) = \{0\} \text{, so ist } \varphi \text{ \textit{injektiv}.}$
\end{lem}

\begin{defi}
  Sobald es einen \textit{Isomorphismus} zwischen $V$ und $W$ gibt,
  so heißen diese zueinander Isomorph. (Notation $V \simeq W$) 
\end{defi}


\begin{lem}
  Sei $V \simeq W$:
  \begin{itemize}
    \item[(a)] Wenn $\varphi(v_{1}), \ldots, \varphi(v_{n}) \in V$ lin. Un., so $v_{1}, \ldots, v_{n} \in W$ ebenfalls lin. Un.
    \item[(b)] $\{ v_{1}, \ldots, v_{n} \} \subset V $ dann Erz. Sys. von $V$, wenn $\{ \varphi(v_{1}), \ldots, \varphi(v_{n}) \} \subset W$ \indent ebenfalls Erz. Sys. von $W$.
    \item[(c)] $v_{1}, \ldots, v_{n} \in V $ dann eine Basis von $V$, wenn $\varphi(v_{1}), \ldots, \varphi(v_{n}) \in W$ eine Basis von $W$ sind.
  \end{itemize}
\end{lem}


\section{Rangsatz}

\begin{defi}
  Sei $\varphi \in \text{Hom}_{\mathbb{K}}(V,W)$ \vspace*{0.2cm} \\
    \centerline{ $\text{rg}(\varphi) := \text{dim}_{\mathbb{K}}(\text{Im}(\varphi))$ }
\end{defi}

\begin{rem}
  Sei $\{ v_{1}, \ldots, v_{n} \}$ eine Basis des $\mathbb{K}$-Vektorraumes $V$, \\ so ist $\text{dim}_{\mathbb{K}}(V) := |\{ v_{1}, \ldots, v_{n} \}| = n \in \mathbb{N}_{0}$
\end{rem}    


  
  




\end{document}
